\documentclass{article}
\usepackage{amsmath}

\begin{document}
\title{Physics Assignment}
\author{Karyampudi Meghana Sai\\ EE23BTECH11031}
\maketitle

\section*{Problem Statement}
Write the first five terms of the sequence \(a_n = \frac{n(n^2+5)}{4}\).

\section*{Solution}
The sequence \(x(n) = \frac{n(n^2+5)}{4}\) starting from \(n = 0\) is:

\begin{align*}
x(0) &= \frac{0(0^2+5)}{4} = \frac{0}{4} = 0 \\
x(1) &= \frac{1(1^2+5)}{4} = \frac{6}{4} = 1.5 \\
x(2) &= \frac{2(2^2+5)}{4} = \frac{18}{4} = 4.5 \\
x(3) &= \frac{3(3^2+5)}{4} = \frac{42}{4} = 10.5 \\
x(4) &= \frac{4(4^2+5)}{4} = \frac{84}{4} = 21 \\
x(5) &= \frac{5(5^2+5)}{4} = \frac{150}{4} = 37.5 \\
\end{align*}


The relation between x(n) and u(n):
\begin{align}
 x(n) = \left(\frac{n^3+5n}{4}\right) \cdot u(n)
 \end{align}
Finding the z transform of u(n):
\[
U(z) = \mathcal{Z}\{u(n)\} = \sum_{n=0}^{\infty} u(n)z^{-n}
\]

Given that the unit step function \(u(n)\) is:

\[ u(n) = \begin{cases} 0 & \text{if } n < 0 \\ 1 & \text{if } n \geq 0 \end{cases} \]

Its Z-transform becomes:

\begin{align}
U(z) &= \sum_{n=0}^{\infty} z^{-n} = 1 + z^{-1} + z^{-2} + z^{-3} + \dotsb \nonumber \\
&= \frac{1}{1 - z^{-1}} \nonumber \\
U(z) &= \frac{1}{1- z^{-1}}
\end{align}
ROC for the z transform of u(n):\\
\text{ROC for } U(z):$\lvert z \rvert > 1$



The $z$ transform of $nu(n)$ can be derived as follows:

Given the signal $nu(n)$, its $z$ transform is represented as:

\[
\mathcal{Z}\{nu(n)\} = \sum_{n=0}^{\infty} nu(n)z^{-n}
\]

The signal $nu(n)$ means the product of $n$ and the unit step function $u(n)$:

\[
nu(n) = n \cdot u(n)
\]

The $z$ transform of $u(n)$ is $U(z) = \frac{z}{z-1}$. To find the $z$ transform of $nu(n)$, let's derive it step by step:

\begin{align*}
\mathcal{Z}\{nu(n)\} &= \sum_{n=0}^{\infty} nu(n)z^{-n} \\
&= \sum_{n=0}^{\infty} n \cdot u(n)z^{-n}
\end{align*}

Now, this expression involves the convolution property of the $z$ transform. The $z$ transform of $n$ is derived separately as:

\begin{align}
\mathcal{Z}\{n\} = \sum_{n=0}^{\infty} nz^{-n-1} = \frac{z^{-1}}{(1-z^{-1})^2}
\end{align}

By convolving the $z$ transform of $n$ with $U(z)$, we get:

\[
\mathcal{Z}\{nu(n)\} = U(z) * \mathcal{Z}\{n\}
\]

Therefore, after convolution:

\[
\mathcal{Z}\{nu(n)\} = U(z) \cdot \mathcal{Z}\{n\} = \frac{1}{1-z^{-1}} \cdot \frac{z^{-1}}{(1-z^{-1})^2}
\]

Simplifying the expression:

\begin{align}
\mathcal{Z}\{nu(n)\} = \frac{z^{-1}}{(1-z^{-1})^3}
\end{align}


Given: $x(n) = n^3u(n)$

The $z$ transform 0f $n^3u(n)$ can be derived as follows:

\begin{align}
\mathcal{Z}\{n^3u(n)\} = \sum_{n=0}^{\infty} n^3u(n)z^{-n} \end{align}

Now, $n^3$ can be represented as a signal $n^3$ convolved with the unit step function $u(n)$:

\[
 n^3u(n) = n^3 * u(n) 
 \]

Using the property that $\mathcal{Z}\{f(n) * g(n)\} = F(z)G(z)$:

\[ \mathcal{Z}\{n^3 * u(n)\} = \mathcal{Z}\{n^3\} \cdot \mathcal{Z}\{u(n)\} \]

The $z$ transform of $n^3$ is found to be:

\begin{align}
 \mathcal{Z}\{n^3\} = \sum_{n=0}^{\infty} n^3z^{-n-1} = \frac{(z^{-1})(1+z^{-1})(1+2z^{-1})}{(1-z^{-1})^4}
  \end{align}



Therefore, the $z$ transform of $n^3u(n)$ can be obtained by multiplying the $z$ transforms of $n^3$ and $u(n)$:

\begin{align}
\mathcal{Z}\{n^3u(n)\} &= \mathcal{Z}\{n^3\} \cdot \mathcal{Z}\{u(n)\} \nonumber\\
&= \frac{(z^{-1})(1+z^{-1})(1+2z^{-1})}{(1-z^{-1})^4} \frac{1}{1-z^{-1}}\nonumber \\
\mathcal{Z}\{n^3u(n)\}&= \frac{(z^{-1})(1+z^{-1})(1+2z^{-1})}{(1-z^{-1})^5} 
\end{align}

Z-Transform of x(n):
\begin{align}
X(z) &=  \frac{\mathcal{Z}\{n^3u(n)\}}{4}+ \frac{\mathcal{Z}\{5nu(n)\}}{4}\\
X(z)&= \frac{(z^{-1})(1+z^{-1})(1+2z^{-1})}{4(1-z^{-1})^5}+ \frac{5z^{-1}}{4(1-z^{-1})^3}
\end{align}

\end{document}

