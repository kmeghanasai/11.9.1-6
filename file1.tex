\documentclass[12pt]{article}
\usepackage{amsmath}

\begin{document}
\title{Maths Assignment}
\author{Karyampudi Meghana Sai\\
        EE23BTECH11031}
\maketitle
\section*{Problem Statement}
Write the first five terms of the sequence \(x(n) = \frac{n(n^2+5)}{4}\).
\section*{Solution}
Given \(x(n) = \frac{n(n^2+5)}{4}\).
\[
\begin{aligned}
\text{First term:}\\\quad &x(1) = \frac{1(1^2 + 5)}{4} = \frac{6}{4} = 1.5 \\
\text{Second term:}\\\quad &x(2) = \frac{2(2^2 + 5)}{4} = \frac{18}{4} = 4.5 \\
\text{Third term:}\\\quad &x(3) = \frac{3(3^2 + 5)}{4} = \frac{42}{4} = 10.5 \\
\text{Fourth term:}\\\quad &x(4) = \frac{4(4^2 + 5)}{4} = \frac{84}{4} = 21 \\
\text{Fifth term:}\\\quad &x(5) = \frac{5(5^2 + 5)}{4} = \frac{150}{4} = 37.5 \\
\end{aligned}
\]

Therefore, the first five terms of the sequence \(x(n)\) are 1.5, 4.5, 10.5, 21, 37.5.


\[x(n) = \frac{n(n^2+5)}{4}\]
\[u(n) = n^2\]

To express \(x(n)\) in terms of \(u(n)\), let's consider the relation \(x(n) = a \cdot u(n) + b\).

Substituting the sequences:
\[\frac{n(n^2+5)}{4} = a \cdot n^2 + b\]

Comparing coefficients:
Coefficient of \(n^2\): \(a = \frac{1}{4}\)
Constant term: \(b = 5a = 5 \cdot \frac{1}{4} = \frac{5}{4}\)

Therefore, relation between \(x(n)\) and \(u(n)\) is:
\[x(n) = \frac{1}{4} \cdot u(n) + \frac{5}{4}\]

If \(u(z) = Z^2\), let's find \(X(z)\) in terms of \(U(z)\).

\[U(z) = Z^2\]
\[X(z) = \frac{1}{4} \cdot U(z) + \frac{5}{4} \cdot \mathcal{Z}\{1\}\]

Applying the Z-transform to \(1\) gives: \(\mathcal{Z}\{1\} = \frac{z}{z-1}\)

Therefore:
\[X(z) = \frac{1}{4} \cdot Z^2 + \frac{5}{4} \cdot \frac{z}{z-1}\]

This expression of \(X(z)\) in terms of \(U(z)\) represents the relation between sequences \(x(n)\) and \(u(n)\), transformed into the \(z\)-domain.
\end{document}

